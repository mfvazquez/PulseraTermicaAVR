\documentclass[10pt,spanish,a4paper,openany,notitlepage]{article}
%-------------------------------------Paquetes-----------------------------------------------------------------------
\usepackage[spanish,es-tabla]{babel}  	% Traduce los textos a castellano
\usepackage[utf8]{inputenc}	% Permite escribir directamente áéíóúñ
\usepackage{t1enc}            	% Agrega caracteres extendidos al font
\usepackage{amsmath} 		%Permite imprimir mas opcciones matematicas
\usepackage{graphicx}		%Permite agregar imagenes al informe
\usepackage{multicol}  		%Permite dividir el texto en varias columnas
\usepackage{float} 		%Permite utilizar H para colocar las imagenes en un lugar especifico 
\usepackage{units}
\usepackage{circuitikz}
\usepackage{caption}
\usepackage{subcaption}
\usepackage{sidecap}
\usepackage{mathtools}
\usepackage{multirow} % Paquete para dividir las tablas en subtablas
\usepackage{booktabs} %estos 2 sirven para achicar la tabla
\usepackage{tabulary}
\usepackage{fancyhdr} % encabezado
\usepackage{textcomp} % para usar ° con el comando \textdegree
\usepackage{anysize}		%Permite modificar los margenes del documento
\usepackage{abstract} % paquete para el resumen del articulo
\usepackage{amssymb}
%---------------------------------------Configuraciones de pagina----------------------------------------------
\marginsize{2.5cm}{2.5cm}{1cm}{1cm}

\pagestyle{fancy}
\fancyhf{}
\lhead{
66.09 - \textsc{Laboratorio de Microcomputadoras}\\ 
1\textsuperscript{er} Cuatrimestre de 2015
}
\rhead{\includegraphics[width=3cm]{imagenes/FIUBA_ALTA.jpg}}
\rfoot{Página \thepage}

%---------------------------------------Definiciones propias---------------------------------------------------------
\newcommand{\oiint}{\displaystyle\bigcirc\!\!\!\!\!\!\!\!\int\!\!\!\!\!\int} %Integral doble cerrada

\DeclarePairedDelimiter\abs{\lvert}{\rvert}%
\DeclarePairedDelimiter\norm{\lVert}{\rVert}%
% Swap the definition of \abs* and \norm*, so that \abs
% and \norm resizes the size of the brackets, and the 
% starred version does not.
\makeatletter
\let\oldabs\abs
\def\abs{\@ifstar{\oldabs}{\oldabs*}}
%
\let\oldnorm\norm
\def\norm{\@ifstar{\oldnorm}{\oldnorm*}}
\makeatother
%--------------------------------------------------------------------------------------------------------------------------------


\makeatletter
\let\ps@plain\ps@fancy 
\makeatother

% lo siguiente es para borrar el titulo del resumen y que no ocupe espacio:
 \AtBeginDocument{%
 \renewcommand{\abstractname}{}%
 }
\renewcommand{\absnamepos}{empty} % originally center
 

\begin{document}
\title{\textbf{Proyecto: Diseño y construcción de una pulsera térmica}}
\author{
  Martinez, Gaston - 91383\\
  \texttt{gaston.martinez.90@gmail.com}  
  \and
   Vázquez, Matías - 91523\\
  \texttt{mfvazquez@gmail.com}
}
\date{5 de abril de 2015}
\maketitle

\begin{abstract} %Resumen
\emph{Se diseñará e implementará una pulsera térmica que regulará la
temperatura corporal. Se utilizará un módulo termoeléctrico para enviar
variaciones de calor o frío a la muñeca del usuario para modificar
la percepción térmica del cuerpo.}
\end{abstract}

\section{Especificaciones}

El dispositivo utilizará una placa de peltier para enviar pulsos de calor
o frío. De forma que se logre una diferencia de temperatura de $0,4\, \unit{^oC/seg.}$
durante 5 segundos y durante los siguientes 10 segundos entrará
en estado de espera, para luego volver a iniciar el ciclo. 

Deberá contar con un sensor de temperatura para medir la temperatura ambiente
y analizar si deberá enviar o recibir calor.

Finalmente deberá controlar que se cumpla el ciclo utilizando una termocupla
para medir la temperatura corporal cercana a la placa de peltier.

\end{document}
